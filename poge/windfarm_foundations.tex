%%%%%%%%%%%%%%%%%%%%%%%%%%%%%%%%%%%%%%%%%
% Simple Sectioned Essay Template
% LaTeX Template
%
% This template has been downloaded from:
% http://www.latextemplates.com
%
% Note:
% The \lipsum[#] commands throughout this template generate dummy text
% to fill the template out. These commands should all be removed when 
% writing essay content.
%
%%%%%%%%%%%%%%%%%%%%%%%%%%%%%%%%%%%%%%%%%

%----------------------------------------------------------------------------------------
% PACKAGES AND OTHER DOCUMENT CONFIGURATIONS
%----------------------------------------------------------------------------------------

\documentclass[12pt]{article} % Default font size is 12pt, it can be changed here
\usepackage[utf8]{inputenc}    % utf8 support       %!!!!!!!!!!!!!!!!!!!!
\usepackage[T1]{fontenc}

\usepackage{geometry} % Required to change the page size to A4
\geometry{a4paper} % Set the page size to be A4 as opposed to the default US Letter

\usepackage{graphicx} % Required for including pictures

\usepackage{float} % Allows putting an [H] in \begin{figure} to specify the exact location of the figure
\usepackage{wrapfig} % Allows in-line images such as the example fish picture

\usepackage[style=numeric, backend=biber, sorting=none]{biblatex}
\bibliography{windfarm_foundations.bib}

\linespread{1.2} % Line spacing

%\setlength\parindent{0pt} % Uncomment to remove all indentation from paragraphs

\graphicspath{{img/}} % Specifies the directory where pictures are stored

\newlength{\wideitemsep}
\setlength{\wideitemsep}{.5\itemsep}
\addtolength{\wideitemsep}{-7pt}
\let\olditem\item
\renewcommand{\item}{\setlength{\itemsep}{\wideitemsep}\olditem}

\begin{document}

%----------------------------------------------------------------------------------------
% TITLE PAGE
%----------------------------------------------------------------------------------------

\begin{titlepage}
  \newcommand{\HRule}{\rule{\linewidth}{0.5mm}} % Defines a new command for the horizontal lines, change thickness here

  \center % Center everything on the page

  \textsc{\LARGE University College Cork}\\[1.5cm] % Name of your university/college
  \textsc{\Large Practical Offshore Geological Exploration}\\[0.5cm] % Major heading such as course name
  % \textsc{\large Minor Heading}\\[0.5cm] % Minor heading such as course title

  \HRule \\[0.4cm]
  { \huge Geological controls on offshore windfarm foundations and engineering solutions}\\[0.5cm] % Title of your document
  \HRule \\[1.5cm]

  \begin{minipage}{0.4\textwidth}
  \begin{flushleft} \large
  \emph{Author:} Peter \textsc{Armstrong} \\% Your name 
  \emph{Student ID:} 115224113
  \end{flushleft}
  \end{minipage}
  ~
  \begin{minipage}{0.4\textwidth}
  \begin{flushright} \large
  % \emph{Supervisor:} \\
  % Dr. James \textsc{Smith} % Supervisor's Name
  \end{flushright}
  \end{minipage}\\[4cm]

  {\large \today}\\[3cm] % Date, change the \today to a set date if you want to be precise

  %\includegraphics{Logo}\\[1cm] % Include a department/university logo - this will require the graphicx package

  \vfill % Fill the rest of the page with whitespace

% itemize changes
\newlength{\wideitemsep}
\setlength{\wideitemsep}{.5\itemsep}
\addtolength{\wideitemsep}{-7pt}
\let\olditem\item
\renewcommand{\item}{\setlength{\itemsep}{\wideitemsep}\olditem}

\end{titlepage}

%----------------------------------------------------------------------------------------
% TABLE OF CONTENTS
%----------------------------------------------------------------------------------------

% \tableofcontents % Include a table of contents

% \newpage % Begins the essay on a new page instead of on the same page as the table of contents 

%----------------------------------------------------------------------------------------
% INTRODUCTION
%----------------------------------------------------------------------------------------

\section{Introduction} % Major section
Offshore wind farms are becoming a significant business in oceans around the world.
Although many companies are developing floating wind turbines, to date, most wind turbines are built on foundations on the seabed.
These foundations need to be robust and long lasting. The ocean is a very inhospitable environment and they need to withstand major loads applied by wind, waves and sea currents. Offshore wind turbines typically have a lifespan of 20 years \cite{Siemens:1}. Offshore wind turbines are increasing in size, so foundation technology must also progress to enable these large machines.

There are many factors that influence foundation design but the geological considerations we will discuss here are:
  \begin{itemize}
    \item Seabed Bathymetry
    \item Seabed soils
    \item Sediment movement
  \end{itemize}

\section{Geological Issues}

\subsection{Seabed Bathymetry}
Depth of the sea is one of the most important considerations for foundation choice. Some foundations may be impractical or unsuitable for deeper waters, and sediment movement will occur at different rates depending on water depth.

The gradient or slope of the seabed is also an issue in offshore wind farming. Seabeds with a gradient above 5\% should be avoided \cite{antonio_pantaleo_feasibility_2005}.

\subsection{Seabed Soils}
The seabed is typically covered with a layer of sediment. This sediment layer is generally thicker near coasts and contains finer grained particles further from shore.
In some areas, there may be little or no sedimentation due to strong currents. Areas like this are less likely be used for offshore wind farming due to the difficulty of construction in this type of environment.

The sediment can consist of calcareous (made from the remains of shells and skeletons), or siliceous particles. Siliceous sediments contain hard, round grains, while calcareous sediments are softer and highly compressible.

The consolidation of sediment is a measure of how much water is in the sediment. Sediment that has been laid down gradually over time and not disturbed is referred to as normally consolidated. Water between sediment particles is gradually removed due to the weight of the sediment above.
Looser, more recent sediment is referred to as under consolidated.

In some areas, the sediment layer has been compressed by the load of a glacier. When the glacier melts, the load is removed. The soil is then referred to as overconsolidated. 

The consolidation of the soil will affect it's load bearing ability.

The sediment present in much of the North Sea - an area with a lot of offshore wind farms - is overconsolidated and dense, with a recent layer of softer material \cite{randolph_offshore_2011}. The overconsolidated soils occur because this area was once covered by glaciers.

The choice and specifications of foundations will differ depending on the consolidation and thickness of sediment layers.
Before selecting a foundation design, it is important to get a three dimensional understanding of the seabed. Boring and penetration tests are used to get a sample of the sediment layers. The results are combined with sonar and bathymetric mapping to build a picture of the site \cite{thornton}.

\subsection{Sediment movement}
The sediment layer on the seabed is constantly changing. Local hydrodynamic effects and more general sediment migrations will affect wind turbines in two major ways. Firstly, the changing level of the seabed will change the effective length of the wind turbine tower. This will change the natural frequency of the turbine tower and could affect the rate of fatigue of the materials.

The other issue is the power connection between the turbine tower and the on-site grid. The connection is typically done using a J-shaped connector coming out of the foundation \cite{OffshoreWind:1}. The connector and electricity cable should be buried in the sediment. If the level of the seabed falls, this cable could be left hanging in open water where it is susceptible to damage.

\subsubsection{Scour}
Sediment scour is erosion of the seabed around structures. It is an issue for many installations in marine or river environments. 

Scour occurs at the base of a structure where a hole or depression in the seabed around the structure is formed. It occurs in moving water when sediment is washed away from around the structure.
The level of scour will depend on the sediment type and the energy in the water.

Whitehouse \cite{whitehouse} states that scour is considerable in structures in sandy areas but not a significant problem for structures placed directly on a dense clay seabed.

The energy in the water will depend on a number of factors. In an offshore setting, water movement and therefore scour can be caused both by tidal currents and wave action.
As the movement of water caused by waves decreases by depth, scour caused by wave action will typically be less in deeper waters. 
Dahlberg \cite{dahlberg} observed that scour is much more of an issue in shallower waters.

The presence of the structure in the flow will affect the energy in the water around the structure. Hydrodynamic calculations will show a water pressure and velocity differentiation around a structure in a flowing liquid. Turbulence may be introduced due to the the increased velocity.
The velocity of the water around the structure can increase to higher than the threshold of sediment motion. When this happens, the sediment is washed away.
According to Shields \cite{shields_application_1936}, the threshold of sediment motion is proportional to the density and the diameter of the sediment particles.


\subsubsection{Sand Waves}
Sand waves are an overall movement of the seabed. Sand wave crests of up to 1 metre can move slowly over time \cite{byrneocleirigh}. This means the seabed level can vary over time and this must be taken into account when designing wind turbine foundations.
Areas with significant sand wave movements may not be suitable for offshore wind farms.


\section{Engineering Solutions}

\subsection{Choice of foundation type}

The two main foundation types used for offshore wind turbines to date are monopile foundations and gravity foundations.

\paragraph{Monopile:} A monopile is a hollow steel tube with a diameter 4m - 8m. It is driven into the seabed to depths of up to 30m using a piledriver or hydraulic hammer. A transition section with an access platform is then installed. This connects the monopile to the turbine tower.

Monopile foundations are the most commonly used foundation and are used in around 75\% of offshore turbines \cite{EWEA_deep_water}. They are commonly used where the seabed is hard or semi-hard and in water depths up to 35m \cite{4Coffshore:1}.

In conditions as found in the North sea and around UK coasts, monopiles are driven through the top layer of poorly consolidated soil to the stiffer over consolidated layers which can support the turbine loads \cite{jf368cam.ac.uk_research}.

If the seabed is hard rock with little sedimentation, it may be necessary to drill a hole for the monopile. It is then grouted into place with concrete.

The like many sites in Europe, the London Array wind turbines use monopile foundations \cite{london_array}.

Monopiles are relatively economic and simple to install compared to other foundations and are the most commonly used foundations for this reason.
In deeper waters (30m +), longer pile sections would result in larger moment forces applied to the foundations \cite{doi:10.1680/ener.11.00003}. This would require monopiles with thicker walls to be driven into the seabed to deeper levels.
Alternative options need to be looked at for these deeper waters.


\paragraph{Gravity Base Foundations:} Gravity base foundations (GBF) are the second most commonly used foundation type \cite{EWEA_deep_water}. They are typically a hollow concrete cassion which can be filled with rocks and sediment to provide sufficient ballast to secure the turbine tower.

If a GBF foundation is to be used, the seabed must be levelled and prepared to support the loads required. The top layer of loose sediment is removed by dredging. The area is then filled with a number of layers of crushed rock to give a level and stable base for the foundation. The foundation is then carefully lowered into place. The sediment removed previously is then used used to fill the hollow sections of the gravity base \cite{ruiz}. 

GBF foundations were chosen for the Thornton Bank wind farm in Belgian waters due to the soil conditions at the site \cite{thornton}. A dense, load bearing layer of sand was found 28m below the seabed. Less competent soils were found on top. If a monopile structure was used here, it would need to extend through the less competent soils to the dense soil below. The dimensions would need to be very large and this meant a monopile structure was not economically viable at this site. Instead, the top layer of sand was dredged to reduce the level by 7m. It was then filled with a number of layers of gravel to create a load bearing foundation. Conical GBF were then installed on the site.

\paragraph{Other Foundation Types:} Jacket, tripod and suction foundations are options that are being developed for deeper waters and bigger wind turbines. They have not yet been deployed at a commercial scale in the wind industry. As the wind industry moves farther offshore, these foundations will become more common. 

As the main reason to choose these foundation types is the loads on the tower rather than seabed geological concerns, they will not be discussed further here.

\subsection{Scour Protection}
Scour protection is an important consideration in any underwater construction.
Due to hydrodynamic effects, scour depth is typically greater around structures with a square cross section than cylindrical structures. Dahlberg \cite{dahlberg} studies GBF structures in the north sea. The Frigg TP1 structure has a square cross section and experiences deep scour of up to 2m at two corners. The nearby TCP2 structure has a cylindrical cross section and does not experience the same level of scour.
Cylindrical cross sections are now generally used for both monopiles and gravity foundations.

\paragraph{}
Scour occurs when the energy in the water is higher than the threshold of sediment motion.
According to Shields \cite{shields_application_1936}, the threshold of sediment motion is proportional to the density and the diameter of the sediment particles.

The most common way to prevent scour is by layering crushed rock and gravel around the base of a structure. The larger material will have a higher threshold of sediment motion and will not get washed away.

There are also scour prevention products on the market which can be installed around turbine foundations. One such product uses a mat of recycled car tyres. The voids in the mat fill with seabed material and the walls of the tyres prevent the material from being washed away \cite{tyres}.

\paragraph{Scour around monopiles:}
Monopiles can extend up to 30m into the seabed. In scour prone sites , if no protection is added around the pile, a scour hole depth of around 1.5 times the pile diameter is expected.
This scour can be accommodated by the designs - increasing the wall thickness and depth of the pile penetration will strengthen the foundation\cite{zaaijer}. This will increase the mass and cost of the pile, but this cost may be justified as scour protection is not needed. The cost benefit analysis of installing scour protection vs designing for scouring would depend on the specific site.

\paragraph{Scour around GBF:}
As gravity foundations sit on the seabed and do not extend deeply through the sediment layers, any level of scour is a major issue.
If the sediment surrounding the foundation is scoured away, the foundation may move or tilt.

The level of scour around a gravity foundation will depend on the geometry of the foundation. Both plan and elevation cross sections will affect the hydrodynamics around the foundation \cite{ruiz}. Tank testing and computational modelling of the design should be completed to estimate the level of scour protection required.

Most gravity foundation designs will require some level of scour protection and these costs must be considered when planning a project.

\paragraph{Dealing with scour:}
Remedial works can add protection after scouring has occurred by adding crushed rock layers around the site. This will be significantly more expensive than installing adequate protection in the construction phase as it requires hiring vessels.

%------------------------------------------------

\section{Conclusion} % Major section
A detailed exploration of the prospective site is required before designing a wind turbine foundation. Bathymetry surveys are very important to discover the basic parameters. Detailed sediment samples are required to understand the nature of the soils. 
Once the nature of the soils has been discovered, the foundation type and level of scour protection required can be determined.

%----------------------------------------------------------------------------------------
% BIBLIOGRAPHY
%----------------------------------------------------------------------------------------
\printbibliography

%----------------------------------------------------------------------------------------

\end{document}