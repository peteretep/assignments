%%%%%%%%%%%%%%%%%%%%%%%%%%%%%%%%%%%%%%%%%
% Simple Sectioned Essay Template
% LaTeX Template
%
% This template has been downloaded from:
% http://www.latextemplates.com
%
% Note:
% The \lipsum[#] commands throughout this template generate dummy text
% to fill the template out. These commands should all be removed when 
% writing essay content.
%
%%%%%%%%%%%%%%%%%%%%%%%%%%%%%%%%%%%%%%%%%

%----------------------------------------------------------------------------------------
%	PACKAGES AND OTHER DOCUMENT CONFIGURATIONS
%----------------------------------------------------------------------------------------

\documentclass[12pt]{article} % Default font size is 12pt, it can be changed here

\usepackage{geometry} % Required to change the page size to A4
\geometry{a4paper} % Set the page size to be A4 as opposed to the default US Letter

\usepackage{graphicx} % Required for including pictures

\usepackage{float} % Allows putting an [H] in \begin{figure} to specify the exact location of the figure
\usepackage{wrapfig} % Allows in-line images such as the example fish picture

\usepackage[style=numeric, backend=biber, sorting=none]{biblatex}
\bibliography{ireland_performance.bib}
\usepackage{url}

\linespread{1.2} % Line spacing

%\setlength\parindent{0pt} % Uncomment to remove all indentation from paragraphs

\graphicspath{{Pictures/}} % Specifies the directory where pictures are stored

\newlength{\wideitemsep}
\setlength{\wideitemsep}{.5\itemsep}
\addtolength{\wideitemsep}{-7pt}
\let\olditem\item
\renewcommand{\item}{\setlength{\itemsep}{\wideitemsep}\olditem}

\begin{document}

%----------------------------------------------------------------------------------------
% TITLE PAGE
%----------------------------------------------------------------------------------------

\begin{titlepage}
  \newcommand{\HRule}{\rule{\linewidth}{0.5mm}} % Defines a new command for the horizontal lines, change thickness here

  \center % Center everything on the page

  \textsc{\LARGE University College Cork}\\[1.5cm] % Name of your university/college
  \textsc{\Large Sustainable Energy}\\[0.5cm] % Major heading such as course name
  % \textsc{\large Minor Heading}\\[0.5cm] % Minor heading such as course title

  \HRule \\[0.4cm]
  { \huge Ireland's Progress on Greenhouse Gas and Renewable Energy Targets}\\[0.5cm] % Title of your document
  \HRule \\[1.5cm]

  \begin{minipage}{0.4\textwidth}
  \begin{flushleft} \large
  \emph{Author:} Peter \textsc{Armstrong} \\% Your name 
  \emph{Student ID:} 115224113
  \end{flushleft}
  \end{minipage}
  ~
  \begin{minipage}{0.4\textwidth}
  \begin{flushright} \large
  % \emph{Supervisor:} \\
  % Dr. James \textsc{Smith} % Supervisor's Name
  \end{flushright}
  \end{minipage}\\[4cm]

  {\large \today}\\[3cm] % Date, change the \today to a set date if you want to be precise

  %\includegraphics{Logo}\\[1cm] % Include a department/university logo - this will require the graphicx package

  \vfill % Fill the rest of the page with whitespace
\end{titlepage}

%----------------------------------------------------------------------------------------
% TABLE OF CONTENTS
%----------------------------------------------------------------------------------------

% \tableofcontents % Include a table of contents

% \newpage % Begins the essay on a new page instead of on the same page as the table of contents 

%----------------------------------------------------------------------------------------
% INTRODUCTION
%----------------------------------------------------------------------------------------

\section{A - Ireland's RES Targets} % Major section
Ireland delivered the 2010 national target for renewable electricity 12 months late, failed to meet the 2010 national target for renewable heat and nearly met the revised 2010 national target for renewable transport.

%------------------------------------------------

\subsection{Ireland's 2010 Renewable Electricity Target}
According to the SEAI \cite{SEAI_targets_faq}, Ireland's European target for renewable electricity contribution by 2010 was 13.2\%. The Irish government also specified a more ambitious target of 15\% contribution by 2010.

The percentage of electricity created from renewable sources in Ireland in 2010 was 14.5 \cite{_eurostat}. The country met the EU RES-E target, but not it's own stricter target.
The EU RES-E was in fact met 12 months previously in 2009 when 13.4\% was contributed by RES.
By 2011, the Irish target was also satisfied when 17.3\% of electricity was generated by RES.

Depending on which target is being referred to, Ireland either met it's target 12 months early or 12 months late.

However, it should be noted that there was a 120ktoe overall reduction in the final electricity energy demand between 2008 and 2009 due to the economic recession \cite{_seai_electricity_sector}.


\subsection{Ireland's 2010 Renewable Heat Target}
The Irish target for renewables in heating for 2010 was 5\% \cite{SEAI_targets_faq}.
According to Eurostat \cite{_eurostat}, renewables contributed 4.5\% of heat energy in 2010. The share increased to 5.1\% by 2011.
Ireland missed it's target in 2010 and the share has increased by approximately 0.3\% per year since 2010. At the current rate of increase, Ireland is unlikely to meet the 2020 target of 10\%.

\subsection{Ireland's 2010 Renewable Transport Target}
The original EU RES-T (renewable energy share in transport) target was 5.75\%.  This target was expected to mainly be met by adding biofuels to the fuel mix. The renewable portion of electricity used in electric vehicles also contributes.
Ireland reduced this target to 3\% because of concerns about the ethics and sustainability of biofuel production.

In 2010, the share of renewables in transport was 2.4\%. By 2011, the share had increased to 3.9\%.

Ireland nearly met the target for 2010, and continues to increase the share.
The current focus in renewable transport is to move to electric cars.


\section{B - Primary Energy Equivalent of Non-Combustable RES}

In 2007, electricity generation from wind energy reached 1,953 GWh and the contribution from hydro power was 663 GWh. Ireland’s primary energy requirement (TPER) in 2007 was 16.2 Mtoe. Using the information on the fossil fuel plant mix in the table, calculate the primary energy equivalent of non-combustible (wind plus hydro) renewable energy (in Mtoe and as a percentage) based on each kWh of wind displacing a kWh from the fossil fuel plant mix.


\begin{tabular}{ l c r r r }
  Ireland 2007 & Coal & Peat & Gas & Oil \\
  Fuel Input (TJ) & 52963 & 18045 & 114425 & 15701 \\
  Electricity Generated (GWh) & 5180 & 1791 & 14240 & 1443 \\
\end{tabular}

\paragraph{Assumptions}
\begin{itemize}
  \item No renewable combustibles (Biomass)
  \item Fuel mix stays the same 
\end{itemize}

81554 TJ non renewable electricity takes 201134 TJ primary fuel to generate.
This gives us the efficiency of whole combustible electricity generation as 40\%.

9417 TJ of renewable electricity is generated. If generated from combustible sources, at the same efficiency as the rest of the system, it would take 23226 TJ of primary fuel.

23226 TJ is 0.55 MToe.
If renewable energy sources were removed, the total primary fuel needed would increase by 28\%.


%----------------------------------------------------------------------------------------
% BIBLIOGRAPHY
%----------------------------------------------------------------------------------------
\printbibliography

%----------------------------------------------------------------------------------------

\end{document}